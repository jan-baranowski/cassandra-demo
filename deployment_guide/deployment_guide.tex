\documentclass{article} %======================================================

% package includes
\usepackage[polish]{babel}  %polish typhographical rules
\usepackage[utf8]{inputenc} %input in in UTF-8
\usepackage{polski}         %more polish in this document
\usepackage[T1]{fontenc}    %font encoding
\usepackage{url}            %URL formatting
\usepackage{xcolor}         %some color definitions
\usepackage{indentfirst}    %indent at the beginning of each paragraph
\usepackage{graphicx}       %image insertion package
\usepackage{listings}       %for shell commands listing

%use some polish typhographical rules borrowed from french
\frenchspacing

%redefine page geometry (margins)
\addtolength{\textwidth}{3cm}
\addtolength{\hoffset}{-1.5cm}
\addtolength{\textheight}{3cm}
\addtolength{\voffset}{-1.5cm}

%define style for bash command listing
\lstdefinestyle{bash}{
    language=bash,                         %listings are in bash
    basicstyle=\small\sffamily,            %basic text style
    stringstyle=\ttfamily,                 %typewriter type for strings
    numbers=none,                          %no line numbers
%    numberstyle=\tiny,
%    numbersep=3pt,
    showstringspaces=false,                %no special characters instead of spaces
    keywordstyle=,%\color{black}\bfseries, %normally keywords would be typed in bold, but since listings package determines keywords wrongly, their style is the same as basic style
    frame=tb,                              %top and bottom border
    columns=fullflexible,                  %fixed columns would make listings not fit into page width
    backgroundcolor=\color{gray!20},       %some nice background color
    linewidth=1.0\linewidth,
    xleftmargin=0.0\linewidth
}

\begin{document} %=============================================================

\title{Apache Cassandra 2.0\\vspace{2ex}Przewodnik instalacji i konfiguracji w systemie Debian Wheezy}
\author{Jan Baranowski, Michał Kaik\\Politechnika Poznańska}

\maketitle

\section{Wstęp}\label{sec:intro}

Niniejszy dokument ma za zadanie przedstawić proces instalacji i~konfiguracji serwera baz danych NoSQL \emph{Apache Cassandra} w~środowisku rozproszonym.
Na potrzeby demonstracji zakłada się że środowisko to będzie składać się z kilku węzłów połączonych siecią lokalną.

\emph{Apache Cassandra} jest serwerem baz danych NoSQL, początkowo rozwijanym przez \emph{Facebooka} na potrzeby umożliwienia efektywnego wyszukiwania wiadomości w~skrzynce odbiorczej.
Obecnie (od marca 2009 roku) \emph{Cassandra} jest rozwijana przez \emph{Apache Foundation} (jako jeden z~projektów top-level) i~stanowi podstawę dla zestawu narzędzi DataStax. 

\emph{Cassandra} powstała jako narzędzie mające w~założeniu cechować się:
\begin{itemize}
\item wysoką dostępnością (\emph{Cassandra} jest określana jako zawsze zapisywalna baza danych)
\item niskim opóźnieniem wykonywanych operacji (\textit{ang.}~latency)
\item odpornością na awarie (możliwością replikacji danych, brakiem komponentów, których awaria może zdestabilizować system (\textit{ang.}~single points of failure))
\item możliwością regulacji kompromisu pomiędzy szybkością działania a~odpornością na awarie i~spójnością replik
\item relatywnie prostym modelem danych
\end{itemize}

Zarówno opisanie kolumnowego modelu danych wykorzystywanego w~\emph{Cassandrze}, jak i~mechanizmów, dzięki którym spełnia ona założenia projektowe nie jest celem tego dokumentu.
Autorzy mogą jedynie podać propozycje publikacji, które opisują wspomniane zagadnienia.
Zarówno dla modelu danych, jak i~budowy wewnętrznej \emph{Cassandry} będzie to przede wszystkim dokumentacja udostępniana przez firmę \emph{DataStax} \cite{datastax}.
Kolumnowy model danych, przedstawiony z~perspektywy osób pracujących z~bazami relacyjnymi został doskonale (choć może zbyt obszernie) opisany w~\emph{Cassandra --- The Definitive Guide} \cite{definitiveguide}.
Z~kolei architektura \emph{Cassandry} w~dużym skrócie (choć zapewne w~sposób wystarczający by służyć jako wstęp do tego dokumentu) została przedstawiona w~prezentacji \cite{prezentacja_srds}.

Procedurę przygotowania klastra przedstawiono na przykładzie systemu operacyjnego Debian GNU/Linux 6, ponieważ uchodzi on za jedno z~lepszych rozwiązań dla serwerów.

\section{Wybór dystrybucji}\label{sec:distro}

Na samym początku nowi użytkownicy \emph{Cassandry} stają przed wyborem dystrybucji tego oprogramowania.
Możliwości są dwie: instalacja standardowej wersji dostarczonej przez \emph{Apache Foundation} oraz instalacja platformy \emph{DataStax}~---~zestawu narzędzi obudowujących \emph{Cassandrę} i~dostarczających m.in. funkcje pozwalające na uproszczenie zarządzania klastrem, wizualne monitorowanie, analizę obciążeń węzłów, itp. (por. rysunek \ref{fig:datastax_arch}).

\begin{figure}[h]
\centering
\includegraphics[width=\linewidth]{gfx/datastax}
\caption{architektura \emph{DataStax} w~odniesieniu do \emph{Cassandry}, źródło:~\cite{whydatastax}}
\label{fig:datastax_arch}
\end{figure}

Wybór jest ważny o~tyle, że pomimo wspólnego fundamentu, jakim jest \emph{Cassandra}, procesy instalacji obu dystrybucji nie mają ze sobą nic wspólnego, tj. (według wiedzy autorów) nie da się doinstalować do dystrybucji \emph{Apache} platformy \emph{DataStax}.

W~tym miejscu należy wspomnieć o~twórcach platformy~---~firmie o~zaskakującej nazwie \emph{DataStax}, zajmującej się dostosowaniem \emph{Cassandry} do potrzeb przedsiębiorstw poprzez m.in. rozwój i~testowanie bazy danych, dostarczanie narzędzi ułatwiających administrację systemem, organizację szkoleń, certyfikację personelu technicznego, itd.
Bodaj najbardziej znaczącym wkładem \emph{DataStax} w~rozwój projektu jest opracowanie szeregu konektorów dla różnych języków programowania, dzięki którym programiści mogą korzystać z~\emph{Cassandry} w~sposób analogiczny do rozwiązań relacyjnych (np. tak jak w~przypadku \emph{MySQL Connectors}, por. \cite{mysql_connectors}) i~rozbudowa dokumentacji technicznej platformy, włączając w~to dokumentację plików konfiguracyjnych, architektury \emph{Cassandry} i języka \emph{CQL} (odpowiednika \emph{SQL} dla kolumnowych baz danych).

Na potrzeby niniejszego dokumentu założono, że właściwym wyborem będzie dystrybucja \emph{Apache Cassandra}, ze względu na to, że dokument ma pełnić rolę wprowadzenia do technologii, nie zaś pozwalać na błyskawiczne wdrożenie systemu (\textit{ang.}~rapid deployment).
Poza tym, przewodniki instalacji i~konfiguracji \emph{DataStaxa} dostępne są np. na stronie \cite{datastax_guides}.

\section{Instalacja}\label{sec:install}

\emph{Cassandra} zostanie zainstalowana przy użyciu narzędzia \emph{APT} z~repozytorium \emph{Apache Software Foundation}.
Wymagane będzie dodanie odpowiedniego repozytorium do listy źródeł \emph{APTa}.
Dodatkowo ze względu na to, że \emph{Cassandra} napisana jest w~Javie, pokazany zostanie proces instalacji \emph{Oracle JRE} w~systemie \emph{Debian} (ta maszyna wirtualna jest zalecana przez twórców \emph{Cassandry}).

\subsection{Wybór maszyny wirtualnej Java}\label{subsec:install_vm}

\emph{Apache Software Foundation} dostarcza pakiety zawierające \emph{Cassandrę} w~formacie \texttt{*.deb}, a~także repozytorium dla \emph{APTa}.
Jedną z~zależności pakietu \texttt{cassandra} jest pakiet \texttt{openjdk-7-jre}.
Jest to w~pewnym sensie sprzeczne z~zaleceniami twórców bazy danych, ponieważ ci rekomendują maszynę wirtualną \emph{Oracle} jako środowisko uruchomieniowe (nawet w~testowanej wersji bazy 2.0 odpowiedni komunikat wyświetlany jest w~logu).
Nie zmienia to faktu, że testy (funkcjonalne, lecz nie wydajnościowe) przeprowadzone na potrzeby tego artykułu udowodniły, że \emph{Cassandra} działa bez zarzutu także w~środowisku \emph{OpenJDK}. 

Należy zatem wybrać pomiędzy stosowaniem się do zaleceń a~prostotą instalacji.

\subsection{Instalacja i konfiguracja Oracje Java}\label{subsec:install_oracle}

Pakiet z~\emph{OpenJDK} jest instalowany jako pakiet zależny podczas instalacji Cassandry.
Jeżeli jednak administrator zdecyduje się użyć \emph{Oracle JVM}, poniżej przedstawiona jest skrótowa procedura instalacji tego oprogramowania w~systemie \emph{Debian}.
Stanowi ona kompilację najprostszych rozwiązań i~tym różni się od większości przewodników dostępnych w~internecie, że poza konsolą systemu nie wymaga żadnych dodatkowych narzędzi (np. mechanizmu transferu plików z~maszyny-terminala do maszyny-serwera).

Pierwszym krokiem jest pobranie pakietu oprogramowania (JRE, nie JDK) ze strony \emph{Oracle}.
Standardowo by pobrać plik należy zaakceptować umowę licencyjną, jednak przy odpowiedniej konfiguracji możliwe jest ominięcie tego kroku \cite{downloading_oracle_java}:

\begin{lstlisting}[style=bash, caption={pobieranie \emph{Oracle JRE}}]
$ wget --no-cookies \
> --no-check-certificate \
> --header "Cookie: oraclelicense=accept-securebackup-cookie" \
> "http://download.oracle.com/otn-pub/java/jdk/7u60-b19/jre-7u60-linux-i586.tar.gz" \
> -O /tmp/jre-7u60-linux-i586.tar.gz
\end{lstlisting}

Popularnym sposobem instalacji pobranego oprogramowania jest wypakowanie archiwum (zazwyczaj do katalogu \texttt{/opt}) i~ręczna konfiguracja ścieżki systemowej (por. \cite{downloading_oracle_java}).
\emph{Debian} dostarcza jednak narzędzie pozwalające przekonwertować archiwum do pakietu \emph{DEB} \cite{installing_oracle_java_on_debian}.

Należy je zainstalować, a~potem użyć:

\begin{lstlisting}[style=bash, caption={budowa pakietu DEB z~\emph{Oracle JRE}}]
$ apt-get install java-package

$ fakeroot make-jpkg /tmp/jre-7u60-linux-i586.tar.gz
\end{lstlisting}

Powstały w~ten sposób pakiet \emph{DEB} należy zainstalować.
Po instalacji należy ustawić \emph{Oracle JVM} jako domyślną maszynę wirtualną w~systemie.

\begin{lstlisting}[style=bash, caption={instalacja i konfiguracja \emph{Oracle JRE}}]
$ dpkg -i /tmp/oracle-j2re1.7_1.7.0+update60_i386.deb

$ update-alternatives --config java
There are 2 choices for the alternative java (providing /usr/bin/java).

  Selection    Path                                           Priority   Status
------------------------------------------------------------
  0            /usr/lib/jvm/java-7-openjdk-i386/jre/bin/java   1051      auto mode
* 1            /usr/lib/jvm/j2re1.7-oracle/bin/java            316       manual mode
  2            /usr/lib/jvm/java-7-openjdk-i386/jre/bin/java   1051      manual mode

Press enter to keep the current choice[*], or type selection number:

$ java -version
java version "1.7.0_60"
Java(TM) SE Runtime Environment (build 1.7.0_60-b19)
Java HotSpot(TM) Client VM (build 24.60-b09, mixed mode)
\end{lstlisting}

Według wiedzy autorów Cassandra nie wymaga ustawiania zmiennej \lstinline[style=bash]!JAVA_HOME! dla żadnego z użytkowników (ani roota, ani użytkownika \textit{cassandra}, właściciela demona). Jeżeli jednak zajdzie taka potrzeba, zainstalowane maszyny wirtualne można znaleźć w katalogu /usr/lib/jvm.

\begin{lstlisting}[style=bash, caption={ustawianie JAVA\textunderscore HOME}]
$ echo "export JAVA_HOME=/usr/lib/jvm/j2re1.7-oracle/" >> /home/<user>/.bashrc
\end{lstlisting}

\subsection{Konfiguracja repozytorium}\label{subsec:install_repo}

By móc zainstalować \emph{Cassandrę}, należy dodać odpowiednie repozytorium \emph{Apache Software Foundation} do źródeł programu \emph{APT}.
Zgodnie z~konwencją każda większa wersja \emph{Cassandry} znajduje się w~osobnym repozytorium.
Na potrzeby tego dokumentu w systemie testowym zostanie zainstalowana \emph{Cassandra 2.0.8}.

Do pliku \texttt{/etc/apt/sources.list} należy dopisać:

\begin{lstlisting}[style=bash, caption={nowe źródła pakietów dla \emph{APTa}}]
deb http://www.apache.org/dist/cassandra/debian 20x main
deb-src http://www.apache.org/dist/cassandra/debian 20x main
\end{lstlisting}

Próba pobrania spisu zawartości repozytorium zakończy się niepowodzeniem, ponieważ \emph{APT} nie zna kluczy publicznych dla repozytorium \emph{Apache Software Foundation}.
Te należy dodać w~następujący sposób:

\begin{lstlisting}[style=bash, caption={pobieranie kluczy publicznych repozytorium \emph{ASF}}]
$ gpg --keyserver pgp.mit.edu --recv-keys F758CE318D77295D
$ gpg --export --armor F758CE318D77295D | sudo apt-key add -

$ gpg --keyserver pgp.mit.edu --recv-keys 2B5C1B00
$ gpg --export --armor 2B5C1B00 | sudo apt-key add -
\end{lstlisting}

Po zakończeniu wszystkich operacji należy pobrać spis zawartości repozytoriów:
\begin{lstlisting}[style=bash, caption={odświeżanie list pakietów}]
$ apt-get update
\end{lstlisting}

\subsection{Instalacja Cassandry}\label{subsec:install_install}

Ostatnim krokiem procesu instalacji jest zainstalowanie \emph{Cassandry} z~użyciem \emph{APTa}:
\begin{lstlisting}[style=bash, caption={instalacja \emph{Cassandry}}]
$ apt-get install cassandra
\end{lstlisting}

W~efekcie \emph{Cassandra} powinna zostać zainstalowana i~uruchomiona.

\section{Konfiguracja węzła}\label{sec:config}

Ta część dokumentu opisuje czynności związane z~konfiguracją węzła \emph{Cassandry} tak, by był on zdolny dołączyć do klastra.
\emph{Cassandra} jest systemem P2P, więc o~utworzeniu klastra węzły decydują wspólnie w~oparciu o~taką samą nazwę klastra zdefiniowaną w~pliku konfiguracyjnym i~włączony kanał komunikacji poprzez plotkowanie.
Kanał ten jest domyślnie wyłączony, zatem każdy węzeł \emph{Cassandry} w~sieci będzie z~początku tworzył osobny klaster.

\subsection{Stan systemu po instalacji}\label{subsec:config_postinst}

Zaraz po instalacji w~systemie pojawia się nowa usługa: \texttt{cassandra}.
Jest ona domyślnie uruchomiona.
Jej konfiguracja zapobiega możliwości dołączenia węzła do jakiegokolwiek klastra ze względu na zablokowanie mechanizmu plotkowania (dokładnie: agent implementujący algorytm plotkujący nasłuchuje wyłącznie na adresie IP 127.0.0.1).

Pliki konfiguracyjne można znaleźć w~katalogu \texttt{/etc/cassandra}.
Są wśród nich:

\begin{itemize}
\item \texttt{/etc/cassandra/cassandra-env.sh}~---~skrypt definiujący zmienne środowiskowe w formie argumentów dla maszyny wirtualnej Java (np. wielkość stosu, ale także m.in. włączenie/wyłączenie mechanizmu \emph{JMX}).
\item \texttt{/etc/cassandra/cassandra-rackdc.properties}~---~informacja o tym, w~którym fizycznie racku i~centrum danych znajduje się obecny węzeł.
\item \texttt{/etc/cassandra/cassandra-topology.properties}~---~przybliżone informacje o tym, w~których rackach i~centrach danych znajdują się inne węzły klastra. Plik ten jest wykorzystywany przez jedną z~kilku wersji algorytmu rozmieszczania replik (\emph{snitcha}).
\item \texttt{/etc/cassandra/cassandra-topology.yaml}~---~plik analogiczny do poprzedniego, wykorzystywany jednak przez inną wersję algorytmu.
\item \texttt{/etc/cassandra/cassandra.yaml}~---~główny plik konfiguracyjny \emph{Cassandry}.
\end{itemize}

Pliki danych (w~tym: commit logi) znaleźć można w~katalogu \texttt{/var/lib/cassandra}.
Czyszcząc zawartość trzech znajdujących się w~nim podkatalogów można zresetować stan klastra.
W~celu wykonania tej czynności autorzy nie zalecają jednak ich całkowitego usuwania (podczas testowania mogą one zostać utworzone na nowo z~prawem zapisu tylko dla roota, co uniemożliwi normalny start usługi).

Plik logu znaleźć można w~\texttt{/var/log/cassandra/system.log}.

\subsection{Jak debugować konfigurację?}\label{subsec:config_debug}

Przed rozpoczęciem wprowadzania zmian do plików konfiguracyjnych zaleca się wyłączenie automatycznego uruchamiania usługi.
Błędnie skonfigurowany węzeł może dołączyć do nieodpowiedniego klastra, a~jego usunięcie w~takim przypadku nie jest zadaniem trywialnym.
Efekt można osiągnąć wykonując polecenie:

\begin{lstlisting}[style=bash, caption={wyłączenie uruchamiania \emph{Cassandry} na jej domyślnych runlevelach}]
$ insserv -r cassandra
\end{lstlisting}

Najprostszym sposobem sprawdzenia poprawności konfiguracji jest uruchomienie węzła w~trybie jawnym (w~przeciwieństwie do deamona).
Służy do tego podane poniżej polecenie, które wypisze log działania węzła na standardowe wyjście.

\begin{lstlisting}[style=bash, caption={testowe uruchamianie \emph{Cassandry}}]
$ cassandra -f  # -f for foreground
\end{lstlisting}

Dodatkowo można zarchiwizować taki log poleceniem \texttt{tee plik.log}, które skopiuje standardowe wejście na standardowe wyjście i do podanego pliku:

\begin{lstlisting}[style=bash, caption={testowe uruchamianie \emph{Cassandry} z archiwizacją logu}]
$ cassandra -f | tee /tmp/cassandra_testrun.log
\end{lstlisting}

\bigskip

\noindent\textbf{UWAGA!} Jeżeli \emph{Cassandra} zostanie uruchomiona w~trybie jawnym podczas gdy równolegle usługa działa w~tle, zamiast informacji o~błędzie w~logu tej pierwszej pojawi się informacja o~wyjątku \texttt{NullPointerException}.

\bigskip

\noindent\textbf{UWAGA!} Jeżeli pierwsze uruchomienie \emph{Cassandry} odbędzie się w~trybie jawnym z~poziomu użytkownika innego niż \texttt{cassandra}, usługa nie będzie miała prawa zapisu do nowo utworzonych katalogów danych, nie będzie więc w~stanie się uruchomić.

\bigskip

\noindent\textbf{UWAGA!} Jeżeli zmienna \texttt{JAVA\textunderscore HOME} dla użytkownika uruchamiającego polecenie \texttt{cassandra~-f} jest ustawiona, ale prowadzi do niepoprawnie zainstalowanej maszyny wirtualnej, \emph{Cassandra} nie uruchomi się, ale ani skrypt startowy ani log nie poinformują o~błędzie.

\subsection{Pliki konfiguracyjne}\label{subsec:config_files}

Role podstawowych plików konfiguracyjnych zostały przedstawione w~punkcie \ref{subsec:config_postinst}.
Tutaj (w~tabeli \ref{tab:config_options}) zostaną opisane parametry konfiguracyjne (z \texttt{/etc/cassandra/cassandra.yaml}), których wartości powinny zostać świadomie dobrane przed uruchomieniem pojedynczego węzła.

Konfigurację węzła ułatwia sam plik konfiguracyjny \emph{Cassandry}, zawierający obszerne komentarze.

\begin{table}[hhh]
\caption{podstawowe parametry konfiguracyjne \emph{Cassandry} (w~\texttt{/etc/cassandra/cassandra.yaml}).}
\begin{tabular}{|r|c|p{7.5cm}|}
\hline 
\textbf{Parametr} & \textbf{Linia} & \textbf{Komentarz}\\
\hline
\hline
\texttt{cluster\_name} & 10 & nazwa klastra, którego ten węzeł jest członkiem\\
\hline
\texttt{seed\_provider/parameters/seeds} & 192 & Węzeł \emph{Cassandry} nie odkrywa innych węzłów automatycznie, stąd dla algorytmu plotkowania wymagana jest lista początkowych "punktów kontaktowych". Oczywiście punkty powinny być tak dobrane by nie dopuścić do partycjonowania klastra.\\
\hline
\texttt{listen\_address} & 297 & Adres IP kanału komunikacyjnego pomiędzy węzłami. Agent algorytmu plotkującego będzie nasłuchiwał na tym adresie. Jeżeli wartość nie zostanie tutaj podana, \emph{Cassandra} wybierze adres IP localhosta.\\
\hline
\texttt{start\_rpc} & 324 & Czy uruchomić serwer \emph{Thrift RPC}. Wyłączenie serwera RPC uniemożliwi dostęp do węzła przez np. \texttt{cqlsh}.\\
\hline
\texttt{rpc\_address} & 335 & Adres IP na którym ma nasłuchiwać serwer \emph{Thifta}.\\
\hline
\end{tabular} 
\label{tab:config_options}
\end{table}

\subsection{Weryfikacja stanu węzła}\label{subsec:config_check}

Szeroko pojęty stan węzła po konfiguracji można sprawdzić na kilka sposobów.
Przede wszystkim po uruchomieniu usługi należy sprawdzić czy ta faktycznie działa i~gdy tak nie jest, przejrzeć log.

W~następnej kolejności można sprawdzić czy \emph{Cassandra} nasłuchuje na portach zdefiniowanych w~pliku konfiguracyjnym:

\begin{lstlisting}[style=bash, caption={sprawdzanie na których portach nasłuchuje \emph{Cassandra}}]
$ netstat -ln4p  #listening, numeric, IPv4 only, with owner process information
\end{lstlisting}

Kolejnym sposobem jest próba połączenia się z~węzłem poprzez Thrift API.

\begin{lstlisting}[style=bash, caption={dostęp do \emph{Cassandry} przez \emph{Thrift API}.}]
$ cqlsh localhost 9160 #use "quit;" to exit CQL shell
\end{lstlisting}

Wreszcie można wyświetlić stan klastra, do którego należy węzeł:

\begin{lstlisting}[style=bash, caption={sprawdzanie stanu klastra}]
$ nodetool --host localhost -p 7199 status # 7199 is the JMX port
Datacenter: datacenter1
=======================
Status=Up/Down
|/ State=Normal/Leaving/Joining/Moving
--  Address    Load       Tokens  Owns (effective)  Host ID                               Rack
UN  127.0.0.1  98.76 KB   256     100.0%            d85efd66-29fb-46d7-b824-a4cc3fc4e75b  rack1
\end{lstlisting}

\section{Konfiguracja klastra}

\subsection{Łączenie węzłów w klaster}

\subsection{Weryfikacja konfiguracji klastra}

\subsection{Działania dodatkowe}

\subsubsection{Synchronizacja zegarów}

\subsubsection{Czyszczenie}

\subsubsection{Konfiguracja uwierzytelniania}

\section{Zarządzanie klastrem}

\subsection{Monitoring węzłów i klastra}

\subsection{Przyłączanie i odłączanie węzłów}

\subsection{Backup danych}

\begin{thebibliography}{10}%===================================================

\bibitem[1]{prezentacja_srds}
Jan Baranowski, Michał Kaik, Maciej Urbański,\\
\emph{Cassandra, Highly available, scalable storage system},\\
Politechnika Poznańska 2014

\bibitem[2]{definitiveguide}
E. Hewitt,\\
\emph{Cassandra, The Defintive Guide},\\
O'Reilly 2010

\bibitem[3]{datastax}
\emph{DataStax},\\
\url{http://www.datastax.com/},\\
dostęp 17 czerwca 2014 r.

\bibitem[4]{downloading_oracle_java}
Daniel Stavrovski,\\
\emph{Installing Oracle JAVA 7 on Debian Wheezy},\\
\url{http://d.stavrovski.net/blog/post/installing-oracle-java-7-on-debian-wheezy},\\
dostęp 16 czerwca 2014 r.

\bibitem[5]{installing_oracle_java_on_debian}
\emph{Java/Sun, Debian Wiki},\\
\url{https://wiki.debian.org/Java/Sun},\\
dostęp 16 czerwca 2014 r.

\bibitem[6]{mysql_connectors}
\emph{MySQL Connectors},\\
\url{http://www.mysql.com/products/connector/},\\
dostęp 16 czerwca 2014 r.

\bibitem[7]{datastax_guides}
\emph{NoSQL Apache Cassandra Documentation},\\
\url{http://planetcassandra.org/documentation/},\\
dostęp 16 czerwca 2014 r.

\bibitem[8]{whydatastax}
\emph{Why DataStax?},\\
\url{http://www.datastax.com/why-datastax}\\
dostęp 16 czerwca 2014 r.

\end{thebibliography}

\end{document}